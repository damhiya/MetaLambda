\documentclass[12pt]{article}

\usepackage[a4paper, total={6in,8in}]{geometry}

\newcommand{\MetaLambda}[0]{MetaLambda }

\newcommand{\Modes}[0]{\textrm{Modes}}
\newcommand{\Types}[0]{\textrm{Types}}
\newcommand{\Ctxs}[0]{\textrm{Contexts}}
\newcommand{\Terms}[0]{\textrm{Terms}}

\newcommand{\upshift}[4]{(#3 \vdash_{#1}^{#2} #4)}
\newcommand{\downshift}[3]{\ \downarrow_{#1}^{#2} #3}
\newcommand{\arrow}[3]{#2 \to_{#1} #3}
\newcommand{\base}[1]{\textbf{base}_{#1}}

\title{\MetaLambda}
\author{SoonWon Moon}

\begin{document}
\maketitle

\MetaLambda is a language for multi-stage programming combining two concepts, contextual modality and adjoint logic.

\section{Syntax}
\[
  \begin{array}{llcl}
    \Modes & a, b & \\
    \Types & A, B & ::= & \upshift{a}{b}{\Psi}{A}
                     \mid \downshift{a}{b}{A}
                     \mid \arrow{a}{A}{B}
                     \mid \base{a}
                     \\
    \Ctxs & \Gamma, \Psi & ::= & \cdot \mid \Gamma, x : A \\
  \end{array}
\]

\end{document}
